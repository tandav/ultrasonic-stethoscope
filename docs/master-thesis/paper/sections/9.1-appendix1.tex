% !TEX root = ../main.tex
\documentclass[../main.tex]{subfiles}
\begin{document}
\section{Приложение 1}
\label{appendix1}
Класс LungsModel для модели распространения звука в легких

\begin{lstlisting}
from pyqtgraph.Qt import QtCore, QtGui
from PyQt5.QtGui import QApplication
import pyqtgraph as pg
import numpy as np
import sys
import signal


class LungsModel():
    l_default = 0.065
    h_default = 0.55e-6
    f_default = 110
    # l_default = 0.1
    # h_default = 1
    # f_default = 440

    def __init__(self, L=l_default, H=h_default, F=f_default):
        # self.r = np.load('../cube-full-460-512-512.npy')
        # self.r = np.load('../cube-full-460-512-512.npy')[::1, ::1, ::1]
        # self.r = np.load('../cube-full-460-512-512.npy')[::2, ::2, ::2]
        # self.r = np.load('../cube-full-460-512-512.npy')[::4, ::4, ::4]
        # k = 11
        # self.r = np.load('../cube-full-460-512-512.npy')[::k, ::k, ::k]
        y_start = 160
        x_start = 145
        sqr = 235
        k = 3
        self.r = np.load('../cube-full-460-512-512.npy')[
            20:80:k,
            y_start : y_start + sqr : k,
            x_start : x_start + sqr : k,
        ]
        # self.r = np.load('../cube-full-460-512-512.npy')[::2, ::2, ::2]
        # self.r = np.random.random((2,3,4))
        self.r[0,0,0] = 3
        self.ro  = 1e-5 + 1.24e-3 * self.r - 2.83e-7 * self.r * self.r + 2.79e-11 * self.r * self.r * self.r
        self.c = (self.ro + 0.112) * 1.38e-6

        self.t = 0
        self.l = L # dt, time step
        self.h = H # dx = dy = dz = 1mm
        self.K = self.l / self.h * self.c
        self.K2 = self.K**2
        self.K_2_by_3 = self.K**2 / 3


        # initial conditions
        self.P_pp = np.zeros_like(self.ro) # previous previous t - 2
        self.P_p  = np.zeros_like(self.ro) # previous          t - 1
        self.P    = np.zeros_like(self.ro) # current           t

        N = self.P.shape[1]
        self.A, self.B, self.C = 2, N//2, N//2 # sound source location
        # self.A, self.B, self.C = 2, N//2, N//2 # sound source location
        # self.oA, self.oB, self.oC = 6, N//2, N//2 # sound source location
        self.oA, self.oB, self.oC = 4, N//2, N//2 # sound source location

        self.f = F

        self.signal_window = 64
        self.source_signal = np.zeros(self.signal_window)
        self.observ_signal = np.zeros(self.signal_window)
        
        print(f'init model l={self.l} h={self.h} f={self.f}')

    def update_P(self):
        '''
        mb work with flat and then reshape in return
        norm by now, mb add some more optimisations in future, also cuda
        '''

        S = self.P_p.shape[0]
        N = self.P_p.shape[1]

        self.P[2:-2, 2:-2, 2:-2] = 2 * self.P_p[2:-2, 2:-2, 2:-2] - self.P_pp[2:-2, 2:-2, 2:-2]

        Z = np.zeros_like(self.P_p)
        Z[2:-2, 2:-2, 2:-2] = 22.5 * self.P_p[2:-2, 2:-2, 2:-2]
        
        cell_indeces_flat = np.arange(S * N * N).reshape(S, N, N)[2:-2, 2:-2, 2:-2].ravel().reshape(-1, 1) # vertical vector

        s1_indexes_flat = cell_indeces_flat + np.array([-1, 1, -N, N, -N**2, N**2])      # i±1 j±1 k±1 
        s2_indexes_flat = cell_indeces_flat + np.array([-1, 1, -N, N, -N**2, N**2]) * 2  # i±2 j±2 k±2 
        s1_values = self.P_p.ravel()[s1_indexes_flat] # each row contains 6 neighbors of cell 
        s2_values = self.P_p.ravel()[s2_indexes_flat] # each row contains 6 neighbors of cell 
        s1 = np.sum(s1_values, axis=1) # sum by axis=1 is faster for default order
        s2 = np.sum(s2_values, axis=1)

        Z[2:-2, 2:-2, 2:-2] -=   4 * s1.reshape(S-4, N-4, N-4)
        Z[2:-2, 2:-2, 2:-2] += 1/4 * s2.reshape(S-4, N-4, N-4)

        m1 = np.array([1, -1, -1/8, -1/8])
        m2 = np.array([1, -1])

        s3_V_indexes = cell_indeces_flat + np.array([N**2, -N**2, 2*N**2, -2*N**2])
        s3_V_values = self.P_p.ravel()[s3_V_indexes] * m1 # po idee mozhno za skobki kak to vinesti m1 i m2
        s3_V_sum = np.sum(s3_V_values, axis=1)
        s3_N_indexes = cell_indeces_flat + np.array([N**2, -N**2])
        s3_N_values = self.ro.ravel()[s3_N_indexes] * m2
        s3_N_sum = np.sum(s3_N_values, axis=1)
        s3 = (s3_V_sum * s3_N_sum).reshape(S-4, N-4, N-4)
        
        s4_V_indexes = cell_indeces_flat + np.array([N, -N, 2*N, -2*N])
        s4_V_values = self.P_p.ravel()[s4_V_indexes] * m1
        s4_V_sum = np.sum(s4_V_values, axis=1)
        s4_N_indexes = cell_indeces_flat + np.array([N, -N])
        s4_N_values = self.ro.ravel()[s4_N_indexes] * m2
        s4_N_sum = np.sum(s4_N_values, axis=1)
        s4 = (s4_V_sum * s4_N_sum).reshape(S-4, N-4, N-4)

        s5_V_indexes = cell_indeces_flat + np.array([1, -1, 2, -2])
        s5_V_values = self.P_p.ravel()[s5_V_indexes] * m1
        s5_V_sum = np.sum(s5_V_values, axis=1)
        s5_N_indexes = cell_indeces_flat + np.array([1, -1])
        s5_N_values = self.ro.ravel()[s5_N_indexes] * m2
        s5_N_sum = np.sum(s5_N_values, axis=1)
        s5 = (s5_V_sum * s5_N_sum).reshape(S-4, N-4, N-4)

        Z[2:-2, 2:-2, 2:-2] += (s3 + s4 + s5) * self.ro[2:-2, 2:-2, 2:-2]
        self.P[2:-2, 2:-2, 2:-2] -= Z[2:-2, 2:-2, 2:-2] * self.K_2_by_3[2:-2, 2:-2, 2:-2]
        # self.P[self.ro < 0.1] = 0
      
    def step(self):
        self.P_old = self.P
        self.update_P()
        self.P[self.A, self.B, self.C] = np.sin(2 * np.pi * self.f * self.t) # sound source
        self.P_pp  = self.P_p
        self.P_p   = self.P_old
        

        self.source_signal = np.roll(self.source_signal, -1)
        self.observ_signal = np.roll(self.observ_signal, -1)
        self.source_signal[-1] = self.P[self.A, self.B, self.C]
        self.observ_signal[-1] = self.P[self.oA, self.oB, self.oC]

        self.t += self.l

\end{lstlisting}
\newpage
\end{document}
