% !TEX root = ../main.tex
\documentclass[../main.tex]{subfiles}
\begin{document}
\addcontentsline{toc}{section}{\protect\numberline{}Список использованных источников}
\renewcommand\refname{Список использованных источников}
% \bibliographystyle{ugost2003}
\begin{thebibliography}{}

% STETHOSCOPES
\bibitem{littmann}
Стетоскоп 3M™ Littmann® Electronic Stethoscope Model 3200\\
\url{http://www.littmann.com/3M/en_US/littmann-stethoscopes/products/~/3M-Littmann-Electronic-Stethoscope-Model-3200?N=5932256+8711017+3293188392&rt=rud}

\bibitem{cms-vesd} 
Стетоскоп CMS-VESD SPO2 PR\\
\url{http://www.contecmed.com/index.php?page=shop.product_details&flypage=flypage.tpl&product_id=26&category_id=13&option=com_virtuemart&Itemid=600}

\bibitem{cms-ve} 
Стетоскоп CMS-VE\\
\url{http://www.contecmed.com/index.php?page=shop.product_details&flypage=flypage.tpl&product_id=21&category_id=13&option=com_virtuemart&Itemid=600}

\bibitem{cms-m} 
Стетоскоп CMS-M\\
\url{http://www.contecmed.com/index.php?page=shop.product_details&flypage=flypage.tpl&product_id=27&category_id=13&option=com_virtuemart&Itemid=600}

\bibitem{cardioics} 
Стетоскоп Cardionics E-scope II\\
\url{http://www.cardionics.com/product/clinical-systems/hearing-impaired-e-scope}

\bibitem{thinklabs-one} 
Стетоскоп Thinklabs One\\
\url{http://www.thinklabs.com/one-digital-stethoscope}

\bibitem{eko-core} 
Стетоскоп Eko Core\\
\url{https://ekodevices.com/core/}

\bibitem{eko-duo} 
Стетоскоп Eko Duo\\
\url{https://ekodevices.com/duo/}

\bibitem{due} 
АЦП Arduino Due\\
\url{https://www.arduino.cc/en/Main/ArduinoBoardDue}

\bibitem{BMP280Library}
Библиотека Adafruit BMP280\\
\url{https://github.com/adafruit/Adafruit_BMP280_Library}

\bibitem{op-amp-shop} 
Схема усилителя для микрофона\\
\url{http://full-chip.net/analogovaya-elektronika/70-usilitel-dlya-elektretnogo-mikrofona-s-nizkim-urovnem-shuma-shema.html}

\bibitem{op-amp} 
Руководство пользователя и технические характеристики операционного усилителя MCP6022\\
\url{https://lib.chipdip.ru/291/DOC000291231.pdf}

\bibitem{python} 
Язык программирования Python\\
\url{https://www.python.org/downloads/}

\bibitem{app-due}
Программа для запуска на компьютере для АЦП Arduino Due\\
\url{https://github.com/tandav/ultrasonic-stethoscope/blob/master/arduino/app.py}

\bibitem{cuda} 
Технология Nvidia CUDA\\
\url{http://www.geforce.com/hardware/technology/cuda}

\bibitem{arduino-ide} 
Программа Arduino IDE\\
\url{https://www.arduino.cc/en/Main/Software}

\bibitem{lungs-2d-youtube}
Видео модели распространения звука в легких\\
\url{https://www.youtube.com/watch?v=E6hVebjHmME}

\bibitem{github} 
Исходный код, документация и этапы создания проекта ультразвукового стетоскопа\\
\url{https://github.com/tandav/ultrasonic-stethoscope}

\bibitem{aaa} 
Горшков, Ю. Г. Обработка речевых и акустических биомедецинских сигналов на основе вейвлетов - Идательство Радиотехника, 2017. - 240 с.

\bibitem{bbb} 
Дьяконов, В. П. Вейвлеты. От теории к практике. М.: СОЛОН-Р, 2002. - 400 с.

\bibitem{ccc} 
Чистович Л. А. Физиология речи. Восприятие речи человеком. АН СССР. Л.: Наука, 1976. - 388 с.

\bibitem{ddd} 
Дворянкин С.В. Взаимосвязь цифры и графики, звука и изображения // Открытые системы, 2000. - 93 с.

\bibitem{eee} 
Блаттер К. Вейвлет-анализ. Основы теории. М.: Техносфера, 2006. - 280 с.

\bibitem{fff} 
Галяшина Е.И. Речь под микроскопом // Компьютерра, 1999. - 10 с.

\end{thebibliography}

\end{document}
