% !TEX root = ../paper.tex
\documentclass[../paper.tex]{subfiles}
\begin{document}

Разработанная информационная система состоит из следующих частей:
\begin{enumerate}
  \item Врач
  \item Пациент (диагностируемый)
  \item Микрофон
  \item Усилитель
  \item Аналого цифровой преобразователь
  \item Компьютер
  \item Удаленный сервер для высокопроизводительных вычислений
\end{enumerate}

Для подавления шумов и лучшей передачи звука от сердца, легких и других органов к микрофону присоединяются мембрана и соединительная трубка от аналогового стетоскопа. Сигнал с микрофона подается на усилитель. С усилителя сигнал подается на Аналого цифровой преобразователь (АЦП). Аналого цифровой преобразователь подключается к компьютеру через USB-порт и передает данные программе.

\begin{center}
Краткая схема устройства:\\
\noindent\small{{Микрофон → Усилитель → АЦП → Компьютер}}
\end{center}

\begin{figure}[H]
\centering
\includegraphics[width=14cm]{images/hardware.png}
\caption{Собранный прототип}
\end{figure}

\subsection{Описание микрофона}
В качестве микрофона был выбран SWEN MK-200. \\

\begin{table}[h]
\centering
\label{my-label}
\begin{tabular}{|l|l|}
\hline
Чувствительность, дБ           & -60 ± 3                    \\ \hline
Диапазон частот, Гц            & 50 – 16 000                \\ \hline
Размер микрофонного модуля, мм & 9×7                        \\ \hline
Тип разъема                    & мини-джек Ø 3,5 мм (3 pin) \\ \hline
Длина кабеля, м                & 1,8                        \\ \hline
Вес, г                         & 63                         \\ \hline
\end{tabular}
\caption{Технические характеристики SWEN MK-200}
\end{table}

Микрофон был вынут из стандартного корпуса, чтобы лучше соединиться с трубкой, ведущей к мембране.

\subsection{Описание усилителя}
Усилитель для микрофона был создан самостоятельно в рамках данной работы.

Была выбрана следующая схема усилителя:

\begin{figure}[H]
\centering
\includegraphics[width=14cm]{images/circuit.jpg}
\caption{Схема усилителя сигнала}
\end{figure}

В качестве операционного усилителя был выбран \textbf{MCP6022} от производителя Microchip. Это усилитель типа Rail-to-Rail SO-8.

SOIC или просто SO (small-outline-integrated-circuit), а также SOP (Small-Outline Package) корпус микросхем, предназначенный для поверхностного монтажа, занимающий на печатной плате на 30-50\% меньше площади чем аналогичный корпус DIP, а также имеющий на 50-70\% меньшую толщину. Обычно в обозначении также указывается число выводов.

В данный усилитель встроены High-Pass и Low-Pass фильтры. High-Pass фильтрует частоты сигнала меньше 1Гц. Low-Pass фильтрует частоты выше 100кГц. Меняя конденсатор С3, можно менять частоту среза LowPass фильтра. Усиление схемы зависит от резисторов R3 и R4. На текущий момент усиление составляет порядка 100.

Ниже приводятся параметры операционного усилителя.

\begin{table}[h]
\centering
\label{my-label}
\begin{tabular}{|l|l|}
\hline
Полоса частот                  & 10МГц                      \\ \hline
Уровень шума                   & 8.7 нВ/√Гц                 \\ \hline
Количество каналов             & 2                          \\ \hline
Напряжение питания             & 2.5В --- 5.5В              \\ \hline
Напряжение смещения            & $\pm500\mu V $             \\ \hline
Гармонические искажения        & 0.00053\%                  \\ \hline
Температурный диапазон         & -40°C --- +85°C            \\ \hline
Тип корпуса                    & SO-8                       \\ \hline
\end{tabular}
\caption{Технические характеристики операционного усилителя MCP6022}
\end{table}

\begin{figure}[H]
\centering
\includegraphics[width=8cm]{images/op-amp.png}
\caption{Распиновка операционного усилителя}
\end{figure}

\begin{figure}[H]
\begin{subfigure}{0.5\textwidth}
\includegraphics[width=0.9\linewidth, height=5cm]{images/op-amp-plot1.png} 
\caption{Напряжение смещения --- Температура}
% \label{fig:subim1}
\end{subfigure}
\begin{subfigure}{0.5\textwidth}
\includegraphics[width=0.9\linewidth, height=5cm]{images/op-amp-plot2.png}
\caption{Шум --- Частота}
% \label{fig:subim2}
\end{subfigure}

\vspace{10mm} % vertical space
 
\begin{subfigure}{0.5\textwidth}
\includegraphics[width=0.9\linewidth, height=5cm]{images/op-amp-plot3.png} 
\caption{Шум - Напряжение смещения}
% \label{fig:subim1}
\end{subfigure}
\begin{subfigure}{0.5\textwidth}
\includegraphics[width=0.9\linewidth, height=5cm]{images/op-amp-plot4.png}
\caption{CMRR --- Частота}
% \label{fig:subim2}
\end{subfigure}

\vspace{10mm} %vertical space

\begin{subfigure}{0.5\textwidth}
\includegraphics[width=0.9\linewidth, height=5cm]{images/op-amp-plot5.png} 
\caption{Коэффициент усиления - Частота}
% \label{fig:subim1}
\end{subfigure}
\begin{subfigure}{0.5\textwidth}
\includegraphics[width=0.9\linewidth, height=5cm]{images/op-amp-plot6.png}
\caption{\small{Гармонические искажения --- Вых. напряжение (f=20кГц)}}
% \label{fig:subim2}
\end{subfigure}

\caption{Параметры операционного усилителя}
\label{fig:image2}
\end{figure}

Разработка платы усилителя велась в программе Sprint Layout. Ниже представлен скриншот макета платы из этой программы.

\begin{figure}[H]
\centering
% \includegraphics[width=14cm]{sprint-layout-circuit.png}
\includegraphics[width=\textwidth]{images/sprint-layout-circuit.png}
\caption{Схема печатной платы усилителя}
\end{figure}

Печатная плата изготавливалась при помощи печати на лазерном принтере методом травления текстолита. 

\subsection{АЦП}
Предназначение АЦП – преобразование непрерывных (аналоговых) входных сигналов в цифровую форму для дальнейшей обработки с помощью компьютера.

В качестве аналого цифрового преобразователя были опробованы 3 устройства:
\begin{enumerate}
  \item LCard E14-140M
  \item ЗАО "Руднев-Шиляев" ЛА-н10-12USB
  \item Arduino Due
\end{enumerate}

\subsubsection{LCard E14-140M}

\begin{table}[H]
\centering
\label{my-label}
\begin{tabular}{|l|l|}
                                                                                    \hline
Количество каналов                 & 16 дифференциальных или 32 \\& с "общей землей" \\ \hline
Максимальная частота дискретизации & 200 кГц                                     \\ \hline
Объем буффера памяти               & 64 Кбайт                                    \\ \hline
Разрядность                        & 14бит                                       \\ \hline
Эффективная разрядность            & 13,3 бит (100 кГц, диап. изм 2,5 В.)        \\ \hline
Входное сопротивление              & Не менее 10 МОм                             \\ \hline
Диапазоны входного напряжения      & $\pm10V;\pm2.5V;\pm0.6V;\pm0.15V$           \\ \hline
Синхронизация                      & От внешнего синхросигнала, \\& по уровню аналогового сигнала, \\& внутренняя. Возможна многомодульная.          \\ \hline
Защита входов                      & 30 В (питание вкл.); 10 В (питание \\& выкл. и в режиме suspend)       \\ \hline
\end{tabular}
\caption{Технические характеристики АЦП LCard E14-140M}
\end{table}

При работе с данным АЦП возникли трудности при написании програмного обеспечения. Средства разработки ПО для данного ацп давно не обновлялись. Работа с АЦП плохо документирована производителем и приведено мало примеров использования. Тем не менее, удалось запустить данный АЦП на одном из примеров производителя. Данный пример позволяет записывать данные с АЦП в бинарный файл.

\subsubsection{ЛА-н10-12USB}

\begin{table}[H]
\centering
\label{my-label}
\begin{tabular}{|l|l|}
                                                                      \hline
Число аналоговых входов            & 2                             \\ \hline
Минимальная частота дискретизации  & 1.25МГц                       \\ \hline
Максимальная частота дискретизации & 80МГц                         \\ \hline
Объем буффера памяти               & $2\times10^{19}=524288$       \\ \hline
Разрядность                        & 12бит (4096 значений)         \\ \hline
Входное сопротивление              & 50Ом                          \\ \hline
Разъем                             & BNC                           \\ \hline
Диапазоны входного напряжения      & $\pm2V;\pm1V;\pm0.4V;\pm0.2V$ \\ \hline
Защита по входному напряжению      & $\pm5V$                       \\ \hline
Дифференциальная нелинейность      & $\pm1.2$ МЗР                  \\ \hline
Интегральная нелинейность          & $\pm1.5$ МЗР                  \\ \hline
Ошибка сдвига                      & $\pm0.15\%$                   \\ \hline
Интерфейс                          & USB                           \\ \hline
Потребляемая мощность              & 12В, 0.7А                     \\ \hline
Масса                              & 400г                          \\ \hline
\end{tabular}
\caption{Технические характеристики АЦП ЛА-н10-12USB}
\end{table}

\textbf{Интегральная нелинейность} - отклонение по вертикальной оси точек реальной характеристики от идеальной характеристики преобразования, делящих пополам расстояние по оси абцисс между средними значениями пороговых уровней характеристики преобразования. Измеряется в процентах или  МЗР.

\textbf{Дифференциальная нелинейность} - отклонение разности двух аналоговых сигналов от значения, соответствующего единице МЗР.

Для данного АЦП было написано програмное обеспечение, позволяющее принимать данные с АЦП и визуализировать их в реальном времени.

Проблемы данного АЦП:

Данный АЦП работает в режиме старт-стоп. Другими словами он периодически производит запуск, сбор данных(в буффер) и остановку. Это один полный цикл сбора данных. При этом полезное время - это сбор данных, а старт и стоп - бесполезное.

Время, за которое АЦП совершает полный цикл сбора данных и соотношение полезного и бесполезного времени отличается в зависимости от частоты дискретизации и размера буффера. Были произведены замеры времени для различных значений частоты дискретизации и размера буффера и были составлены следующие таблицы.

Во всех таблицах ось X: размер буффера, ось Y: частота дискр.

\begin{figure}[H]
\centering
% \includegraphics[width=14cm]{sprint-layout-circuit.png}
\includegraphics[width=\textwidth]{images/cycle-time.png}
\caption{Время на полный цикл (мс)}
\end{figure}

\begin{figure}[H]
\centering
% \includegraphics[width=14cm]{sprint-layout-circuit.png}
\includegraphics[width=\textwidth]{images/good-time.png}
\caption{Полезное время (мс)}
\end{figure}

\begin{figure}[H]
\centering
% \includegraphics[width=14cm]{sprint-layout-circuit.png}
\includegraphics[width=\textwidth]{images/bad-div-by-good.png}
\caption{Отношение полезного времени к полному}
\end{figure}

Как видно из таблицы, даже при самых оптимальных значениях размера буффера и частоты дискретизации, отношение полезного времени к полному составляет 77\%. (полезное время 655мс, полное -  851мс) Тоесть в конце каждого цикла образуется дырка длиной 196мс. При таком долгом времени на перезапуск теряется очень много информации, что неприемлимо для данного проекта. Поэтому было принято решение заменить его. Данный же подходит для коротких сигналов (меньше секунды), которые нужно оцифровывать в сверхвысоком качестве.

\subsubsection{Arduino Due}
\begin{table}[H]
\centering
\label{my-label}
\begin{tabular}{|l|l|}
                                                                      \hline
Число аналоговых входов            & 12                            \\ \hline
Максимальная частота дискретизации & 1МГц                          \\ \hline
Объем буффера памяти               & 512 KB					       \\ \hline
Разрядность                        & 12бит (4096 значений)         \\ \hline
Рабочее напряжение                 & 3.3V                          \\ \hline
Диапазоны входного напряжения      & 7-12V                         \\ \hline
Защита по входному напряжению      & 6-16V                         \\ \hline
Интерфейс                          & USB                           \\ \hline
Микроконтроллер                    & AT91SAM3X8E                   \\ \hline
Масса                              & 36г                           \\ \hline
\end{tabular}
\caption{Технические характеристики АЦП Arduino Due}
\end{table}

Самым оптимальным вариантом ацп для данного проекта оказался Arduino Due. Он сочетает в себе как простоту в использовании так и возможность оцифровывать сигнал высокого качества. Максимальная частота дискретизации  Arduino Due составляет 1МГЦ. В ходе данной работы удалось достичь максимума в 672кГц. Обычно среднее значение частоты дискретизации составляло 666кГц.

\end{document}
